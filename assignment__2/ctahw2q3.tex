%% \listfiles
\documentclass[apj]{emulateapj}
%\documentclass[preprint2,12pt]{emulateapj}
%% \usepackage{natbib}
\usepackage{graphicx}
\usepackage{epsfig}
\usepackage{amssymb,amsmath}
\usepackage{array}
\usepackage{threeparttable}


\singlespace

%definitions
\newcommand{\Msol}{${\rm M_{\sun}}$}


%% Editing markup...
\usepackage{color}


%%%%%%%%%%%%%%%%%%%%%%%%%%%%%%%%%%%%%%%%%%%%%%%%%%%%%%%%%%%%%%%%%%%%%%%%%%%
% WARNING: This LaTeX block was generated automatically by authors.py
% Do not change by hand: your changes will be lost.

%%%%%%%%%%%%%%%%%%%%%%%%%%%%%%%%%%%%%%%%%%%%%%%%%%%%%%%%%%%%%%%%%%%%%%%%%%%


% --------------------- Ancillary information ---------------------
\shortauthors{SURP et al.}
\shorttitle{my short-title}
\slugcomment{Draft: \today}


\begin{document}

\title{CTA200 Assignment 2}
 %% ---------
 
\author{Rosalind Liang\altaffilmark{1}}
\altaffiltext{1}{CITA, University of Toronto}
 


\section{Question 1}
\label{sec:Question 1}

In this question, we are asked to study the behaviour of the sequence $z_{i+1}=z_i^2+c$ in the complex plane, where $c = x + yi$ and $-2<x<2,   -2<y<2$. In particular, we seek to 1) distinguish the points on this domain for which the sequence converges and the points on which it diverges; 2) to map the iteration numbers for which each divergence occurs.
\subsection{Q1 methods}
\label{sec:cmb_data}

Part 1: To locate the coordinates for which the sequence diverges, I used a for loop to pass through 100 iterations of the sequence and collected points of divergence with np.nan(z), which recognizes when either the real or imaginary component of z is no longer a recognizable number (suggesting divergence to infinity). Matplotlib was then used to plot these coordinates. 

Part 2: A similar for loop  was used with np.isfinite(z) to count the number of iterations each coordinate took before diverging. For cases where divergence did not occur within 100 iterations, a special value of -20 was assigned to create contrast in the colourmap, before plotting with matplotlib again.

\subsection{Q1 esults}
\label{sec:Q1 results}
The resulting plot of part 1 is shown in Figure 1, where all points of divergence are in yellow and points that remain bounded are in black. The plot for part 2 is shown in Figure 2, where all bounded points are in purple and the rest of the plot is a colourmap of each point's divergence iteration. We observe a fractal shaped zone where the sequence stays bounded, ranging from -1 to 1 in the complex axis and -2 to .5 in the real axis. In figure 2, we see that iterations of divergence are significantly higher around the contours of this zone, and drop to approach zero moving away from it. 

\section{Question 2}
\label{sec:Question 2}

In this question, we are asked to study the SIR model of disease spread. In particular, we seek to integrate and plot the ODE's $dS/dt$,    $dI/dt$   and   $dR/dt$ given in the assignment, and repeat the process with three different sets of the constants $\beta$ and $\gamma$

\subsection{Q2 methods}
\label{sec:Q2 methods}

To integrate each of the ODE's, I defined the model as a function and used odeint from scipy to perform the integration. I then played around with different values of $\beta$ and $\gamma$ before selecting a few which generated interesting plots.

\subsection{Q2 results}
\label{sec:Q2 results}
I first tried setting both $\beta$ and $\gamma$ to 1, to see the raw behaviour of the integrated function (Figure 3). 
I then tried $\beta = .1$ and $\gamma = -.1$ (Figure 4) and $\beta = 3$ and $\gamma = 2$ (Figure 5), all of which produced distinct behaviours between the susceptible, infected and recovered proportions over time. Since a discussion section is not required, I will just let you look at them as I'm very tired and would like to go to bed.

\begin{figure}
\includegraphics[width=1\columnwidth]{q11.png}
\caption{Zone where sequence stays bound is in black, zone of divergence in yellow}
\label{fig:figureOfSpectrum}

\includegraphics[width=1\columnwidth]{q12.png}
\caption{Zone where sequence stays bound is in purple, divergence zone coloured by iteration of divergence}
\label{fig:figureOfSpectrum}
\end{figure}


\begin{figure}
\includegraphics[width=1\columnwidth]{q21.png}
\caption{First set of constants: $\beta=1, \gamma=1$}
\label{fig:figureOfSpectrum}

\includegraphics[width=1\columnwidth]{q22.png}
\caption{First set of constants: $\beta=.1, \gamma=-.1$}
\label{fig:figureOfSpectrum}

\includegraphics[width=1\columnwidth]{q23.png}
\caption{First set of constants: $\beta=3, \gamma=2$}
\label{fig:figureOfSpectrum}
\end{figure}




 


%\acknowledgments

%% %% \bibliographystyle{act}
%% \bibliographystyle{apj}

%% \bibliography{lenscib_refs.bib,apj-jour}



\end{document}
